\documentclass[11pt]{article}
\usepackage{ctex}
\usepackage{mathptmx}
\usepackage[T1]{fontenc}
\usepackage[english]{babel}
\usepackage{blindtext}
\usepackage{nameref}
\usepackage{fancyhdr}
\usepackage{amsmath,amssymb,amsthm}
\usepackage{graphicx,float}
\usepackage{physics}
\usepackage[a4paper, total={6in, 9in}]{geometry}
\usepackage{listings}
\usepackage{color}

\definecolor{dkgreen}{rgb}{0,0.6,0}
\definecolor{gray}{rgb}{0.5,0.5,0.5}
\definecolor{mauve}{rgb}{0.58,0,0.82}

\lstset{frame=tb,
  language=Java,
  aboveskip=3mm,
  belowskip=3mm,
  showstringspaces=false,
  columns=flexible,
  basicstyle={\small\ttfamily},
  numbers=none,
  numberstyle=\tiny\color{gray},
  keywordstyle=\color{blue},
  commentstyle=\color{dkgreen},
  stringstyle=\color{mauve},
  breaklines=true,
  breakatwhitespace=true,
  tabsize=3
}

\graphicspath{ {../Pictures/UML/} }

\pagestyle{plain}
\fancyhf[CF]{\thepage}

\title{Testing Document\\PacMan\\version: 1}
\author{Group F8\\1155127434 HO Chun Lung Terrance\\
Department of Philosophy, The Chinese University of Hong Kong\\1155143519 WOO Pok\\
Department of Physics, The Chinese University of Hong Kong\\1155157839 NG Yu Chun Thomas\\
Department of Computer Science and Engineering, The Chinese University of Hong Kong\\1155157719 LEUNG Kit Lun Jay\\
Department of Computer Science and Engineering, The Chinese University of Hong Kong\\1155143569 MOK Owen\\
Department of Mathematics, The Chinese University of Hong Kong}
\date{\today}

\begin{document}
    \maketitle
    \tableofcontents
    \newpage

    \section{TEST PLAN}
    In order to test the software properly, we will apply a bigbang approach of software testing. We consider for each functionality of the software, there will be pure approaches (black- or white-box testing). To be specific, for complicated functionalities (mostly inter-module functionals and implementational functionals), we choose to apply black-box testing; for simple functionalities (mostly mono-module functionals and object reactions), we choose to apply white-box testing. Afterall, we will integrate the modules and test the whole system directly.

    The testing will be under following environment:
    \begin{itemize}
        \item Software: Unity (version 2021.3.17f1),
        \item Hardware: 
    \end{itemize}

    We list out the functionalities to be tested:

    \subsection*{Database}
    We test the integrity of the Firebase with Unity using white-box testing, as we should see how Firebase responds to user actions.

    \subsection*{User Management}
    We test the Signup function, Login function, and Logout function using mixed-box testing, as we should see how the Login function and Signup function integrate with the database.

    \subsection*{Application}
    \subsubsection*{Non-gameplay items}
    The main window should be displayed, and we test the menu items' functionality, the title screen's functionality, the Gameover screen, and level choosing, including every button on the screen. Every button will be tested using black-box testing.

    \subsubsection*{Image \& arithmetic}
    We test the rendering of images, text, and game status using mixed-box testing, as we wish to see if the rendering works well with these functionalities.

    \subsubsection*{Character's Movement and controls}
    We test the keyboard control and basic movement of the characters by mixed-box testing. We wish to see different implementations yield different results in the gameplay experience.
    
    \section{TEST CASES}
    \subsection*{Character's Movement and controls}
    For character movement control (both the player and the enemies ), the main class of interest is the movement class controlling the player's action. We will be focusing on white box testing on the function SetDirection().

    
    \begin{lstlisting}
    public void SetDirection(Vector3 targetDir, bool forced = false) {
        Debug.Log("SetDirection() called with target direction: " + targetDir.ToString() + " and forced: " + forced.ToString());
    
        if (isBlocked(targetDir) && !forced) {
            nextDirection = targetDir;
            Debug.Log("Path is blocked. Setting nextDirection to: " + nextDirection.ToString());
        }
        else {
            nextDirection = targetDir;
            direction = targetDir;
            nextDirection = targetDir;
            Debug.Log("Path is not blocked. Setting direction and nextDirection to: " + direction.ToString());
    
            rotation = new Vector3(0, Mathf.Atan2(direction.z, -1 * direction.x) * Mathf.Rad2Deg - 90, 0);
            if (rotation.y < 0) rotation += new Vector3(0, 360, 0);
            if (rotation.y >= 360) rotation -= new Vector3(0, 360, 0);
            // Force rotation.y to become 0 or 90 or 180 or 270
            rotation = new Vector3(0, Mathf.Round(rotation.y / 90) * 90, 0);
            Debug.Log("Calculated rotation: " + rotation.ToString());
        }
    
        // Add this debug line to log the current direction, nextDirection, and rotation
        Debug.Log("Current direction: " + direction.ToString() + ", nextDirection: " + nextDirection.ToString() + ", rotation: " + rotation.ToString());
    }
    \end{lstlisting}

    And here is the function update() in movement.cs who take care of the variable NextDirection.

    \begin{lstlisting}
    private void Update() {
        if (nextDirection != Vector3.zero) {
            SetDirection(nextDirection);
        }
    
        // Add this debug line to log the current nextDirection value
        Debug.Log("Current nextDirection: " + nextDirection.ToString());
    }
    \end{lstlisting}
    
    Here we added debug message for checking the validity of the output direction sequence.
    
    \subsubsection*{Inputs and expected outputs}
    \begin{enumerate}
        \item a. Test Case 1: Pacman moves in an open path
        \begin{enumerate}
            \item Input: targetDir = Vector3.forward (0, 0, 1)
            \item Expected Output: direction and nextDirection should be set to Vector3.forward, and the appropriate rotation should be calculated.
        \end{enumerate}
        
        \item b. Test Case 2: Pacman moves in a blocked path without forced movement
        \begin{enumerate}
            \item Input: targetDir = Vector3.left (1, 0, 0) and a wall in the left direction
            \item Expected Output: nextDirection should be set to Vector3.left, but direction should not change.
        \end{enumerate}
        
        \item c. Test Case 3: Pacman moves in a blocked path with forced movement
        \begin{enumerate}
            \item Input: targetDir = Vector3.left (1, 0, 0), a wall in the left direction, and forced = true
            \item Expected Output: direction and nextDirection should be set to Vector3.left, and the appropriate rotation should be calculated.
        \end{enumerate}
        
        \item d. Test Case 4: Pacman changes direction when a previous blocked path becomes unblocked
        \begin{enumerate}
            \item Input: Pacman is moving forward (0, 0, 1) and attempted to turn left (1, 0, 0) but it was blocked. The wall blocking the path is now removed.
            \item Expected Output: direction and nextDirection should be set to Vector3.left, and the appropriate rotation should be calculated.
        \end{enumerate}
    \end{enumerate}

    \subsubsection*{Outputs and analysis}
    Here are the output log messages:
    
    \item a. Test Case 1: Pacman moves in an open path
    \begin{lstlisting}
    SetDirection() called with target direction: (0, 0, 1) and forced: False
    Path is not blocked. Setting direction and nextDirection to: (0, 0, 1)
    Calculated rotation: (0, 0, 0)
    Current direction: (0, 0, 1), nextDirection: (0, 0, 1), rotation: (0, 0, 0)
    \end{lstlisting}

    \item b. Test Case 2: Pacman moves in a blocked path without forced movement
    \begin{lstlisting}
    SetDirection() called with target direction: (1, 0, 0) and forced: False
    Path is blocked. Setting nextDirection to: (1, 0, 0)
    Current direction: (previous direction), nextDirection: (1, 0, 0), rotation: (previous rotation)
    \end{lstlisting}

    \item c. Test Case 3: Pacman moves in a blocked path with forced movement
    \begin{lstlisting}
    SetDirection() called with target direction: (1, 0, 0) and forced: True
    Path is not blocked. Setting direction and nextDirection to: (1, 0, 0)
    Calculated rotation: (0, 270, 0)
    Current direction: (1, 0, 0), nextDirection: (1, 0, 0), rotation: (0, 270, 0)
    \end{lstlisting}

    \item d. Test Case 4: Pacman changes direction when a previous blocked path becomes unblocked
    \begin{lstlisting}
    Current nextDirection: (1, 0, 0)
    SetDirection() called with target direction: (1, 0, 0) and forced: False
    Path is not blocked. Setting direction and nextDirection to: (1, 0, 0)
    Calculated rotation: (0, 270, 0)
    Current direction: (1, 0, 0), nextDirection: (1, 0, 0), rotation: (0, 270, 0)
    \end{lstlisting}

    The test cases are successful. Furthermore, simple black box testing (on modules other than this one, or say the overall movement ability) was also done when testing this current module. We conclude that we did not discover significant faults on the character movement feature.
    

\end{document}