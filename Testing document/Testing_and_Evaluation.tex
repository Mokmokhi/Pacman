\documentclass[11pt]{article}
\usepackage{ctex}
\usepackage{mathptmx}
\usepackage[T1]{fontenc}
\usepackage[english]{babel}
\usepackage{blindtext}
\usepackage{nameref}
\usepackage{fancyhdr}
\usepackage{amsmath,amssymb,amsthm}
\usepackage{graphicx,float}
\usepackage{physics}
\usepackage[a4paper, total={6in, 9in}]{geometry}

\graphicspath{ {../Pictures/UML/} }

\pagestyle{plain}
\fancyhf[CF]{\thepage}

\title{Testing Document\\PacMan\\version: 1}
\author{Group F8\\1155127434 HO Chun Lung Terrance\\
Department of Philosophy, The Chinese University of Hong Kong\\1155143519 WOO Pok\\
Department of Physics, The Chinese University of Hong Kong\\1155157839 NG Yu Chun Thomas\\
Department of Computer Science and Engineering, The Chinese University of Hong Kong\\1155157719 LEUNG Kit Lun Jay\\
Department of Computer Science and Engineering, The Chinese University of Hong Kong\\1155143569 MOK Owen\\
Department of Mathematics, The Chinese University of Hong Kong}
\date{\today}

\begin{document}
    \maketitle
    \tableofcontents
    \newpage

    \section{TEST PLAN}
    In order to test the software properly, we will apply a bigbang approach of software testing. We consider for each functionality of the software, there will be pure approaches (black- or white-box testing) or mixed approach (black- and white-box testing). To be specific, for complicated functionalities (mostly inter-module functionals and implementational functionals), we choose to imply black-box testing; for simple functionalities (mostly mono-module functionals and object reactions), we choose to imply mixed-box testing. Afterall, we will integrate the modules and test the whole function directly.

    The testing will be under following environment:
    \begin{itemize}
        \item Software: Unity (version 2021.3.17f1),
        \item Hardware: 
    \end{itemize}

    We list out the functionalities to be tested:

    \subsection*{Database}
    We test the integrity of the Firebase with Unity using white-box testing, as we should see how Firebase responds to user actions.

    \subsection*{User Management}
    We test the Signup function, Login function, and Logout function using mixed-box testing, as we should see how the Login function and Signup function integrate with the database.

    \subsection*{Application}
    \subsubsection*{Non-gameplay items}
    The main window should be displayed, and we test the menu items' functionality, the title screen's functionality, the Gameover screen, and level choosing, including every button on the screen. Every button will be tested using black-box testing.

    \subsubsection*{Image \& arithmetic}
    We test the rendering of images, text, and game status using mixed-box testing, as we wish to see if the rendering works well with these functionalities.

    \subsubsection*{Character's Movement and control}
    We test the keyboard control, basic movement of Pacman and enemies, and enemies' AI by mixed-box testing. We wish to see different implementations yield different results to the gameplay experience.
    
    \section{TEST CASES}
    \subsection*{Player control and Movement}
    For player control, the main module of interest is the movement.cs module controlling the player's action. Here we add debug message and check the validity of the output direction sequence.
    \subsubsection*{Purpose}
    
    \subsubsection*{Inputs and expected outputs}

\end{document}