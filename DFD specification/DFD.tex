\documentclass{article}
\usepackage{ctex}
\usepackage[english]{babel}
\usepackage{blindtext}
\usepackage{nameref}
\usepackage{fancyhdr}
\usepackage{amsmath,amssymb,amsthm}
\usepackage{graphicx,float}
\usepackage{physics}
\usepackage[a4paper, total={6in, 9in}]{geometry}

\graphicspath{ {../Pictures/} }

\pagestyle{plain}
\fancyhf[CF]{\thepage}

\title{DFD specification Document\\Become Pac-Man\\version: 1}
\author{Group F8\\1155127434 HO Chun Lung Terrance\\
Department of Philosophy, The Chinese University of Hong Kong\\1155143519 WOO Pok\\
Department of Physics, The Chinese University of Hong Kong\\1155157839 NG Yu Chun Thomas\\
Department of Computer Science and Engineering, The Chinese University of Hong Kong\\1155157719 LEUNG Kit Lun Jay\\
Department of Computer Science and Engineering, The Chinese University of Hong Kong\\1155143569 MOK Owen\\
Department of Mathematics, The Chinese University of Hong Kong}
\date{\today}

\begin{document}
\maketitle
\tableofcontents
\newpage

\section{High-Level Context Diagram}

\section{Feature Diagrams}
\subsection{Game Flow}
\subsubsection{Description}
\par The Sign-up function is to create a new datafile for new users of the game, while the Login function is to identify who is accessing the application and provide security check for the user datafile. The game procedure is to integrate the gameplay from level selection and game setting for each user. The setting overwrite function is to manage the in-game experience for users. The shop function is to manage items of a user. The record update function is to update the game record to the database after each gameplay. The show record function is to display statistics of users and the whole game.
\subsubsection{DFD}
\begin{figure}[H]
    \centering
    \includegraphics*[scale=0.4]{gameflow_DFD.png}
\end{figure}
\end{document}