\documentclass[11pt]{article}
\usepackage{ctex}
\usepackage{mathptmx}
\usepackage[T1]{fontenc}
\usepackage[english]{babel}
\usepackage{blindtext}
\usepackage{nameref}
\usepackage{fancyhdr}
\usepackage{amsmath,amssymb,amsthm}
\usepackage{graphicx,float}
\usepackage{physics}
\usepackage[a4paper, total={6in, 9in}]{geometry}

\graphicspath{ {../Pictures/UML/} }

\pagestyle{plain}
\fancyhf[CF]{\thepage}

\title{UML and UI Design Document\\Become Pac-Man\\version: 1}
\author{Group F8\\1155127434 HO Chun Lung Terrance\\
Department of Philosophy, The Chinese University of Hong Kong\\1155143519 WOO Pok\\
Department of Physics, The Chinese University of Hong Kong\\1155157839 NG Yu Chun Thomas\\
Department of Computer Science and Engineering, The Chinese University of Hong Kong\\1155157719 LEUNG Kit Lun Jay\\
Department of Computer Science and Engineering, The Chinese University of Hong Kong\\1155143569 MOK Owen\\
Department of Mathematics, The Chinese University of Hong Kong}
\date{\today}

\begin{document}
\maketitle
\tableofcontents
\newpage
\section{UML DESIGN}

\subsection{User Management}
\subsubsection{Structural Diagram}
\textbf{class diagram}\\
\subsubsection{UMLs}
\textbf{use-case diagram}\\
\textbf{sequence diagram}\\
\subsubsection{Component Functionality}
This component is 
\subsubsection{Major Procedure and Functions}
Signup: This function is to create accounts and save them into database.

Login: This function is to verify account information so that users with records are allowed to retrieve past records in the game.

Verification: This function is to verify the validity of user id and password.

\subsection{Props}
\subsubsection{Structural Diagram}
\textbf{class diagram}\\
\begin{figure}[H]
    \centering
    \includegraphics*[scale=0.4]{Props-Class.png}
\end{figure}
\subsubsection{UMLs}
\textbf{use-case diagram}\\
\begin{figure}[H]
    \centering
    \includegraphics*[scale=0.4]{Props_use-case.png}
\end{figure}
\textbf{sequence diagram}\\
\begin{figure}[H]
    \centering
    \includegraphics*[scale=0.4]{Props_sequence.png}
\end{figure}
\subsubsection{Component Functionality}
The props class is to provide the bases for game objects, such as pellets in the maze for Pacman to achieve scores, or other props that will enhence in-game experience.

The PropsBase component is to provide an abstract template for each prop. And PropsManager is to announce control sequence to each props.
\subsubsection{Major Procedure and Functions}
Use(void): The Use function is to obtain the action of using the prop, by destroying the prop from the maze.

Instantiate(Props): The instantiate function is to create the props for each game.

Destroy(Props): The destroy function is to destroy the prop from the maze for each game.

\subsection{Ghost}
\subsubsection{Structural Diagram}
\textbf{class diagram}\\
\begin{figure}[H]
    \centering
    \includegraphics*[scale=0.25]{Ghost_Class.png}
\end{figure}
\subsubsection{UMLs}
\textbf{activity diagram}\\
\begin{figure}[H]
    \centering
    \includegraphics*[scale=0.3]{Ghost_Activity.png}
\end{figure}
\textbf{sequence diagram}\\
\begin{figure}[H]
    \centering
    \includegraphics*[scale=0.4]{Ghost_Sequence.png}
\end{figure}
\subsubsection{Component Functionality}
The Ghost class is to indicate movement for each of all ghosts in the game. It manipulate with ghost behaviour and different status of a ghost.

It consists of a ghost element from GameManager to control the action of the ghost. Ghost behaviour will be uses to indicate the status of the ghost. GhostChase, GhostScatter, GhostHome and GhostFrightened are 4 status of ghost interacting with GhostBehaviour.
\subsubsection{Major Procedure and Functions}
Enable() and Disable(): The two functions set in Ghost Behaviour provide the indication of activation of a ghost, i.e. whether the ghost is working.

ResetState(): This function is to reset the status of a ghost to its default status.

ExitTransition(): This function is to end the transition procedure.



\section{UI DESIGN}

\subsection{Login page}
\subsubsection{Description of view}
\subsubsection{Screen Image}
    \centering
    \includegraphics*[scale=0.4]{UI0.0Login.png}
\subsubsection{Objects and actions}
The screen in view is the Login screen, and will be the first view the user sees when the program starts. 
Here user can login their account by entering their email and password, and press the "Login" button. Once a user login, they will be brought to the Title Screen
The user can also create an account by pressing the "Sign Up" button, which will bring the user to the sign up screen as follows.

\subsection{Sign-up page}
\subsubsection{Description of view}
\subsubsection{Screen Image}
    \centering
    \includegraphics*[scale=0.4]{UI0.1SignUp.png}
\subsubsection{Objects and actions}
The screen in view is the Sign up screen, it will be opened when the user presses the "Sign Up" button in the Login screen. 
Here user can create an account by inputting their email, password, and confirm passwords, and press the "Sign Up" button.
The user can also go back to the Login screen by pressing the "Go Back" button.

\subsection{Title Screen}
\subsubsection{Description of view}
\subsubsection{Screen Image}
    \centering
    \includegraphics*[scale=0.4]{UI1.0Main.png}
\subsubsection{Objects and actions}



\subsection{Setting UI}
\subsubsection{Description of view}
\subsubsection{Screen Image}
\subsubsection{Objects and actions}

\subsection{Main Game}
\subsubsection{Description of view}
\subsubsection{Screen Image}
\subsubsection{Objects and actions}

\subsection{Record page}
\subsubsection{Description of view}
\subsubsection{Screen Image}
\subsubsection{Objects and actions}

\end{document}