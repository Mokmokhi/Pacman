\documentclass[11pt]{article}
\usepackage{ctex}
\usepackage{mathptmx}
\usepackage[T1]{fontenc}
\usepackage[english]{babel}
\usepackage{blindtext}
\usepackage{nameref}
\usepackage{fancyhdr}
\usepackage{amsmath,amssymb,amsthm}
\usepackage{graphicx,float}
\usepackage{physics}
\usepackage[a4paper, total={6in, 9in}]{geometry}

\graphicspath{ {../Pictures/UML/} }

\pagestyle{plain}
\fancyhf[CF]{\thepage}

\title{UML and UI Design Document\\Become Pac-Man\\version: 1}
\author{Group F8\\1155127434 HO Chun Lung Terrance\\
Department of Philosophy, The Chinese University of Hong Kong\\1155143519 WOO Pok\\
Department of Physics, The Chinese University of Hong Kong\\1155157839 NG Yu Chun Thomas\\
Department of Computer Science and Engineering, The Chinese University of Hong Kong\\1155157719 LEUNG Kit Lun Jay\\
Department of Computer Science and Engineering, The Chinese University of Hong Kong\\1155143569 MOK Owen\\
Department of Mathematics, The Chinese University of Hong Kong}
\date{\today}

\begin{document}
\maketitle
\tableofcontents
\newpage
\section{UML DESIGN}

\subsection{Manager Class}
\subsubsection{Structural Diagram}
\subsubsection{UMLs}
\textbf{use-case diagram}\\
\textbf{class diagram}\\
\begin{figure}[H]
    \centering
    \includegraphics*[scale=0.4]{UML-Class-Singleton.png}
\end{figure}
\textbf{sequence diagram}\\
\subsubsection{Component Functionality}
\subsubsection{Major Procedure and Functions}

\subsection{Props Class}
\subsubsection{Structural Diagram}
\subsubsection{UMLs}
\textbf{use-case diagram}\\
\begin{figure}[H]
    \centering
    \includegraphics*[scale=0.4]{Props_use-case.png}
\end{figure}
\textbf{class diagram}\\
\begin{figure}[H]
    \centering
    \includegraphics*[scale=0.4]{Props-Class.png}
\end{figure}
\textbf{sequence diagram}\\
\begin{figure}[H]
    \centering
    \includegraphics*[scale=0.4]{Props_sequence.png}
\end{figure}
\subsubsection{Component Functionality}
The props class is to provide the bases for game objects, such as pellets in the maze for Pacman to achieve scores, or other props that will enhence in-game experience.
\subsubsection{Major Procedure and Functions}
Use(void): The Use function is to obtain the action of using the prop, by destroying the prop from the maze.

Instantiate(Props): The instantiate function is to create the props for each game.

Destroy(Props): The destroy function is to destroy the prop from the maze for each game.

\subsection{Database}
\subsubsection{Structural Diagram}
\subsubsection{UMLs}
\textbf{use-case diagram}\\
\textbf{class diagram}\\
\textbf{sequence diagram}\\
\subsubsection{Component Functionality}
\subsubsection{Major Procedure and Functions}


\subsection{Ghost Class}
\subsubsection{Structural Diagram}
\subsubsection{UMLs}
\textbf{activity diagram}\\
\begin{figure}[H]
    \centering
    \includegraphics*[scale=0.3]{Ghost_Activity.png}
\end{figure}
\textbf{class diagram}\\
\begin{figure}[H]
    \centering
    \includegraphics*[scale=0.25]{Ghost_Class.png}
\end{figure}
\textbf{sequence diagram}\\
\begin{figure}[H]
    \centering
    \includegraphics*[scale=0.4]{Ghost_Sequence.png}
\end{figure}
\subsubsection{Component Functionality}
\subsubsection{Major Procedure and Functions}



\section{UI DESIGN}

\subsection{Login/Sign-up page}

\subsection{Title Screen}

\subsection{Shop UI}

\subsection{Setting UI}

\subsection{Main Game}

\subsection{Record page}

\end{document}