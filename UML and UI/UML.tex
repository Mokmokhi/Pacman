\documentclass[11pt]{article}
\usepackage{ctex}
\usepackage{mathptmx}
\usepackage[T1]{fontenc}
\usepackage[english]{babel}
\usepackage{blindtext}
\usepackage{nameref}
\usepackage{fancyhdr}
\usepackage{amsmath,amssymb,amsthm}
\usepackage{graphicx,float}
\usepackage{physics}
\usepackage[a4paper, total={6in, 9in}]{geometry}

\graphicspath{ {../Pictures/UML/} }

\pagestyle{plain}
\fancyhf[CF]{\thepage}

\title{UML and UI Design Document\\Become Pac-Man\\version: 1}
\author{Group F8\\1155127434 HO Chun Lung Terrance\\
Department of Philosophy, The Chinese University of Hong Kong\\1155143519 WOO Pok\\
Department of Physics, The Chinese University of Hong Kong\\1155157839 NG Yu Chun Thomas\\
Department of Computer Science and Engineering, The Chinese University of Hong Kong\\1155157719 LEUNG Kit Lun Jay\\
Department of Computer Science and Engineering, The Chinese University of Hong Kong\\1155143569 MOK Owen\\
Department of Mathematics, The Chinese University of Hong Kong}
\date{\today}

\begin{document}
\maketitle
\tableofcontents
\newpage
\section{UML DESIGN}

\subsection{Database Management}
\subsubsection{Structural Diagram}
\begin{figure}[H]
    \centering
    \includegraphics*[scale=0.4]{Database_struct.png}
\end{figure}
\subsubsection{UMLs}
\textbf{class diagram}\\
\begin{figure}[H]
    \centering
    \includegraphics*[scale=0.4]{Database_Class.drawio.png}
\end{figure}
\textbf{use-case diagram}\\
\begin{figure}[H]
    \centering
    \includegraphics*[scale=0.4]{Database_use-case.png}
\end{figure}
\textbf{activity diagram}\\
\begin{figure}[H]
    \centering
    \includegraphics*[scale=0.4]{Database_Activity.png}
\end{figure}
\subsubsection{Component Functionality}
This component is to manage all the manager and user information consisting of the database action. The DataBaseManager controls Register, Login, LeaderBoard and UserData Managers. Each of which are related to its corresponding procedure.
\subsubsection{Major Procedure and Functions}
Register: This function is to create accounts and save them into database.

Login: This function is to verify account information so that users with records are allowed to retrieve past records in the game.

Verification: This function is to verify the validity of user id and password.

\subsection{Props}
\subsubsection{Structural Diagram}
\begin{figure}[H]
    \centering
    \includegraphics*[scale=0.4]{Props_struct.png}
\end{figure}
\subsubsection{UMLs}
\textbf{class diagram}\\
\begin{figure}[H]
    \centering
    \includegraphics*[scale=0.4]{Props-Class.png}
\end{figure}
\textbf{use-case diagram}\\
\begin{figure}[H]
    \centering
    \includegraphics*[scale=0.4]{Props_use-case.png}
\end{figure}
\textbf{sequence diagram}\\
\begin{figure}[H]
    \centering
    \includegraphics*[scale=0.4]{Props_sequence.png}
\end{figure}
\subsubsection{Component Functionality}
The props class is to provide the bases for game objects, such as pellets in the maze for Pacman to achieve scores, or other props that will enhence in-game experience.

The PropsBase component is to provide an abstract template for each prop. And PropsManager is to announce control sequence to each props.
\subsubsection{Major Procedure and Functions}
Use(void): The Use function is to obtain the action of using the prop, by destroying the prop from the maze.

Instantiate(Props): The instantiate function is to create the props for each game.

Destroy(Props): The destroy function is to destroy the prop from the maze for each game.

\subsection{Pacman}
\subsubsection{Structural Diagram}
\begin{figure}[H]
    \centering
    \includegraphics*[scale=0.4]{Pacman_struct.png}
\end{figure}
\subsubsection{UMLs}
\textbf{class diagram}\\
\begin{figure}[H]
    \centering
    \includegraphics*[scale=0.4]{Pacman_Class.png}
\end{figure}
\textbf{use-case diagram}\\
\begin{figure}[H]
    \centering
    \includegraphics*[scale=0.4]{Pacman_use-case.png}
\end{figure}
\textbf{sequence diagram}

Eat Ghost:
\begin{figure}[H]
    \centering
    \includegraphics*[scale=0.4]{Pacman_sequence-eatGhost.png}
\end{figure}
Turn:
\begin{figure}[H]
    \centering
    \includegraphics*[scale=0.4]{Pacman_sequence-turn.png}
\end{figure}
\subsubsection{Component Functionality}
The Pacman will be with a Transform component to produce transformation, a RigidBody component to handle gravity motion, and a collider component to handle collisions.

A pacman script and movement script will be handling the attributes and the motion of the pacman respectively.

\subsubsection{Major Procedure and Functions}
EatPellet(pellet: Pellet): This function handles when a pellet is eaten by the pacman. If it is true, the score will be increased.

EatPowerPellet(powerPellet: Pellet): This function handles when a power pellet is eaten by the pacman. Pacman state will be according to the power pellet eaten.

Eaten(): This function handles the death of Pacman. If a pacman is eaten by a ghost, the lives will be decreased and Pacman will be respawned somewhere.

SetDirection(): This function handles the moving direction of the Pacman. It uses the checkIsBlocked() function to check whether the direction to change is being blocked by some walls. If so, it remains the same direction as before the change.
\subsection{Ghost}
\subsubsection{Structural Diagram}
\begin{figure}[H]
    \centering
    \includegraphics*[scale=0.4]{Ghost_Structural.png}
\end{figure}
\subsubsection{UMLs}
\textbf{class diagram}\\
\begin{figure}[H]
    \centering
    \includegraphics*[scale=0.25]{Ghost_Class.png}
\end{figure}
\textbf{activity diagram}\\
\begin{figure}[H]
    \centering
    \includegraphics*[scale=0.3]{Ghost_Activity.png}
\end{figure}
\textbf{sequence diagram}\\
\begin{figure}[H]
    \centering
    \includegraphics*[scale=0.4]{Ghost_Sequence.png}
\end{figure}
\subsubsection{Component Functionality}
The Ghost class is to indicate movement for each of all ghosts in the game. It manipulate with ghost behaviour and different status of a ghost.

It consists of a ghost element from GameManager to control the action of the ghost. Ghost behaviour will be uses to indicate the status of the ghost. GhostChase, GhostScatter, GhostHome and GhostFrightened are 4 status of ghost interacting with GhostBehaviour.
\subsubsection{Major Procedure and Functions}
Enable() and Disable(): The two functions set in Ghost Behaviour provide the indication of activation of a ghost, i.e. whether the ghost is working.

ResetState(): This function is to reset the status of a ghost to its default status.

ExitTransition(): This function is to end the transition procedure.

\section{UI DESIGN}

\subsection{Login page}
\subsubsection{Description of view}
The screen is shown with a background of character view of game maze, with a yellow text of game title `Pacman 3D'. There will be 2 textboxes for entering Email and Password of the user, and two buttons for Sign up or Login respectively,
\subsubsection{Screen Image}
\begin{figure}[H]
    \centering
    \includegraphics*[scale=0.2]{UI0.0Login.png}
\end{figure}
\subsubsection{Objects and actions}
The screen in view is the Login screen, and will be the first view the user sees when the program starts. 

Here user can login their account by entering their email and password into the provided textboxes, and press the "Login" button. Once user is verified, they will be brought to the Title Screen.

The user can also create a new account by pressing the "Sign Up" button, which will direct the user to the sign up screen as follows.

\subsection{Sign-up page}
\subsubsection{Description of view}
In the sign up page, the background is consistently using the character view in the maze, with a yellow text Sign up indicating the page usage. There will be 3 textboxes for entering email, password, and the confirmation password for double entry verification. A go back button and a sign up button will be at the bottom of the screen.
\subsubsection{Screen Image}
\begin{figure}[H]
    \centering
    \includegraphics*[scale=0.2]{UI0.1SignUp.png}
\end{figure}
\subsubsection{Objects and actions}
The Sign up screen will be opened when the user presses the "Sign Up" button in the Login screen. 

Here user can create an account by inputting their email, password, and confirm passwords, and press the "Sign Up" button for system verification. If Sign up process is failed, there will be a message showing that attempt is unsuccessful.

The user can also go back to the Login screen by pressing the "Go Back" button.

\subsection{Title Screen}
\subsubsection{Description of view}
The title screen is with a character view in the maze, with yellow as a background color for UI. A title `Pacman', and 5 buttons including `Start Game', `Settings', `Score Board', `Login Out' and `Quit to Desktop'. 
\subsubsection{Screen Image}
\begin{figure}[H]
    \centering
    \includegraphics*[scale=0.2]{UI1.0Main.png}
\end{figure}
\subsubsection{Objects and actions}
The button `Start Game' will direct user to the main game, while the button `Setting' will direct user to the setting page. A `Score Board' button allow users to go to the score board page. A `Log Out' button will let user to log out from current account and go to the Login screen, while the button `Quit to Desktop' will do the logout and at the same time leave the application directly.

\subsection{Level Selection Screen}
\subsubsection{Description of view}
The level selection screen is with a character view in the maze, with yellow as a background color for UI. A title `Level Selection', and 3 boxes for difficulty selection: `Easy', `Normal', `Hard', and 3 tags for level selection. A `Back to Menu' button is placed at the bottom of the screen.
\subsubsection{Screen Image}
\begin{figure}[H]
    \centering
    \includegraphics*[scale=0.2]{UI1.1LevelSelection.png}
\end{figure}
\subsubsection{Objects and actions}
The tick box for difficulty selection considered to be the difficulty of ghost AI. The easy difficulty will have the dumbest AI, while hard difficulty will have the smartest AI, nearly a cheated level.

The tags under the level selection provides level selection. The 3 levels are corresponding to 3 different maps.

The `Back to Menu' button will direct users back to title screen.


\subsection{Setting UI}
\subsubsection{Description of view}
The Setting page consists of a character view in maze as backgound, a big yellow text `Settings' indicating this is a setting page, with a textbox with a `Change' button, 3 scroll bars, a cancel button and a done button.
\subsubsection{Screen Image}
\begin{figure}[H]
    \centering
    \includegraphics*[scale=0.2]{UI1.2Settings.png}
\end{figure}
\subsubsection{Objects and actions}
The textbox is used for inputting a new user name, and when the `Change' button is clicked, it means to confirm the change with the inputted name.

The scroll bars will be available for controlling the master volume, the effect volume, the music volume of the game. For user who want to ignore the changes, pressing the cancel button allows him to return previous page discarding the changes made, or he could press the done button to save the changes.

\subsection{Scoreboard}
\subsubsection{Description of view}
The Setting page consists of a character view in maze as backgound, a big yellow text `Leader Board' indicating this is a Leader Board page. A table consisting Rank, User Name, Score. There is a `Go Back' button in the bottom.
\subsubsection{Screen Image}
\begin{figure}[H]
    \centering
    \includegraphics*[scale=0.2]{UI1.3ScoreBoard.png}
\end{figure}
\subsubsection{Objects and actions}
The table will show the ranking over users by listing the rank according to the highest score of the users. For each gameplay, the score of the latest gameplay of current user will be highlighted and be placed on the top of the scoreboard.

When the `Go back' button is clicked, user will be directed to title screen.

\subsection{Main Game}
\subsubsection{Description of view}
In the main gameplay, it will be shown as first person view, with brick walls, pellets and ghosts. There are labels on the top-left corner showing score and lives, and a mini-map in the bottom left corner for a bounded bird-eye view of the maze. 
\subsubsection{Screen Image}
\begin{figure}[H]
    \centering
    \includegraphics*[scale=0.2]{UI2.0InGame.png}
\end{figure}
\subsubsection{Objects and actions}
The score label will show the current score of the player dynamically when user eat any pellets or props increasing the scores. The lives label shows the remaining number of lives of the player. Once it goes 0, it will direct to Game Over screen. If player eats all the pellets for scores, then it will direct to the Game success screen.

Ghosts in the maze will be spawned by time and chases player. Player should escape from their attacks.

The mini-map shows the current position of player and his surroundings, which helps player to find direction but keeping the first person experience.

\subsection{Game Pause Menu}
\subsubsection{Description of view}
The pause screen consists of a character view in maze as background, and yellow text `Paused' indicating this is the pause menu. The menu contains a `resume' button, `Setting' button, `Quit to Menu' button and `Quit to Desktop' button.
\subsubsection{Screen Image}
\begin{figure}[H]
    \centering
    \includegraphics*[scale=0.2]{UI2.1PauseMenu.png}
\end{figure}
\subsubsection{Objects and actions}
The Resume button allows player resume the game. The `Settings' button will direct player to the setting page. `Quit to Menu' button will direct player to title page, and `Quit to Desktop' will directly close the application.
\subsection{Game Over}
\subsubsection{Description of view}
The Game over screen consists of a character view in maze as background, and yellow text `Game over' indicating this is the Game over screen. The menu contains a `Back to menu' button.
\subsubsection{Screen Image}
\begin{figure}[H]
    \centering
    \includegraphics*[scale=0.2]{UI2.2GameOver.png}
\end{figure}
\subsubsection{Objects and actions}
The `Back to menu' button will direct player to the title screen.
\subsection{Game Clear}
\subsubsection{Description of view}
The Game clear screen consists of a character view in maze as background, and yellow text `You Survived' indicating this is the Game clear screen. The menu contains a `Back to menu' button.
\subsubsection{Screen Image}
\begin{figure}[H]
    \centering
    \includegraphics*[scale=0.2]{UI2.3GameSuccess.png}
\end{figure}
\subsubsection{Objects and actions}
The `Back to menu' button will direct player to the title screen.

\end{document}